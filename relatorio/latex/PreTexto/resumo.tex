%%%% RESUMO
%%
%% Apresentação concisa dos pontos relevantes de um texto, fornecendo uma visão rápida e clara do conteúdo e das conclusões do trabalho.

\begin{resumoutfpr}%% Ambiente resumoutfpr
ESTABILIZADOR DE DOIS EIXOS PARA CÂMERAS. 18f. Oficina de Integração 1 - Relatório Final - Curso de Engenharia de Computação, UNIVERSIDADE TECNOLÓGICA FEDERAL DO PARANÁ (UTFPR). Curitiba, 2022.\\\\O presente documento descreve um dispositivo destinado à estabilização de câmeras por meio de uma estrutura articulada. O projeto foi idealizado visando atender uma necessidade específica do ramo audiovisual, a grande maioria das câmeras ainda em atuação no mercado não possuem mecanismos de supressão à vibrações, sendo muito perceptível em suas gravações, trepidações que, rotineiramente, obrigam à realização de novas gravações ou a perda do material gravado devido à baixa qualidade. Com o objetivo de diminuir o problema, foi criado um dispositivo que cancela grandes vibrações utilizando dois motores servo e um micro controlador Arduino nano, tal dispositivo opera sobre dois eixos, alterando a inclinação vertical e horizontal de uma estrutura articulada, que, por fim, estabiliza o dispositivo de vídeo que é anexado a ela.


\end{resumoutfpr}
