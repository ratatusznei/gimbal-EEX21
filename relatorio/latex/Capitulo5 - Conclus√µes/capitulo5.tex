%%%% CAPÍTULO 5 - CONCLUSÕES E PERSPECTIVAS
%%
%% Deve finalizar o trabalho com uma resposta às
%% hipóteses especificadas na introdução. O autor deve
%% manifestar seu ponto de vista sobre os resultados
%% obtidos; não se deve incluir neste capítulo novos
%% dados ou equações. A partir da tese, alguns assuntos
%% que foram identificados como importantes para serem
%% explorados poderão ser sugeridos como temas para
%% novas pesquisas.

%% Título e rótulo de capítulo (rótulos não devem conter caracteres especiais, acentuados ou cedilha)
\chapter{Conclusões}\label{cap:conclusoeseperspectivas}

Apesar das dificuldades encontradas durante a construção e programação do dispositivo, nota-se que, mesmo componentes de prototipação conseguem entregar um resultado aceitável. Este projeto, mesmo sendo conceitualmente simples, se mostrou desafiador devido a possuir partes móveis acopladas, oque a principio, gerou problemas de estabilidade e, no caso da estrutura articulada, retrabalho de algumas peças. Além disso, o uso de ferramentas de edição 3d facilitou muito o trabalho de construí-lo, assim como a visualização 3d das leituras do sensor por meio de um software contribuiu para o entendimento do funcionamento do mesmo.
